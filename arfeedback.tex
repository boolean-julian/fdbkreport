\documentclass[11pt, a4paper]{article} 
\usepackage[utf8]{inputenc}
\usepackage{geometry}
\usepackage{setspace}
\usepackage{biblatex}
\usepackage[hidelinks]{hyperref}
\usepackage[nolist,nohyperlinks]{acronym}
\usepackage{lipsum}

\onehalfspacing
\addbibresource{arfeedback.bib}

% Original assignment: Evaluate how to react on wrong AR usage
\newcommand{\mytitle}{Evaluation of Textual Feedback for Wrong Usage of Augmented Reality Applications}
\newcommand{\runninghead}{Feedback in Augmented Reality}
\newcommand{\myauthor}{Julian Lüken, Mehmed Mustafa, Jan Schneider,\\ Steffen Tunkel, Chris Warin}
\newcommand{\myuni}{Georg-August University, Göttingen}
\newcommand{\titlespace}{1em}

\markright{\uppercase{\runninghead}\hfill}
\newenvironment{myabstract}{\begin{abstract}\begin{itshape}}{\end{itshape}\end{abstract}}

\begin{document}
	% Acronyms go here, refer to them with \ac{shorthand}
	\begin{acronym}
		\acro{AR}{Augmented Reality}
	\end{acronym}

	\newgeometry{left=25mm,right=25mm,top=30mm,bottom=30mm}
	\pagestyle{empty}
	\begin{center}
		\begin{minipage}{.8\textwidth}
			\centering
			\begin{doublespace}\huge\textbf{\mytitle}\normalsize\\[\titlespace]\end{doublespace}
			\textsc{\myauthor}\\[\titlespace]
			\today\\[\titlespace]
			\myuni\\[\titlespace]
		\end{minipage}
	\end{center}
	\vspace{\titlespace}
	\begin{myabstract}
		% Abstract goes here
		\lipsum[1-3]
	\end{myabstract}
	\newgeometry{left=25mm,right=25mm,top=30mm,bottom=30mm}
	\pagestyle{myheadings}
	\pagebreak

	% REMEMBER
	%
	% Write in active, not passive
	% Write as "we"
	% Only two tenses (present and past)
	% Use LaTeX acronym environment (use \ac{shorthand} and define them right after \begin{document})
	% No paragraphs with only one sentence
	% No different types of paragraphs for pseudo-structuring
	% No forward references
	% Cite consistently and clearly (\cite{Lastname1999}, Jabref)
	% Footnotes: Just where really required
	% No repetitions

	\section*{Introduction}
		% Motivation
		% * Growing use of Augmented Reality and Lack of good feedback
		% * new technology: people don't yet know how to use AR features
		% * making AR usable without huge introductory efforts
		%
		% Research goals/questions
		% * To provide appropriate feedback to "wrong" AR usage/gestures
		% 	* To decide what type of feedback is better in comparison to other types
		%
		% Structure of report
		% * "For validation of our approach we conducted a usability case study"
		% * What will follow in the next sections
		%
		\lipsum[1]

	\section*{Foundations} 
		% * What is AR?
		% * Which frameworks/technologies were used?
		% 	* Brief info about Vivian (which relies on state machines)
		% 		* What gestures are "correct"? (vivian relies on 1 finger shit and moving environment)
		%
		% * What is Usability(-testing)?
		% 	* Explain SUS (how did we tailor it to serve our purposes?)
		%
		\lipsum[2]

	\section*{Related Work}
		% Papers (as to be found in *.bib):
		% * Dey2016: A Systematic Review of 10 Years of Augmented Reality Usability Studies: 2005 to 2014
		% * Nilsson2007: Fun and Usable: Augmented Reality Instructions in a Hospital Setting
		% * Poupyrev2002: Developing a Generic Augmented Reality Interface
		%
		% parallels between above papers and this one
		% where do we come into play in above papers?
		% establish link between recent and past research
		%
		\lipsum[3]

	\section*{Approach}
		% What are "wrong" inputs? Which ones are "wrong"?
		% Describe feedback messages (size, color, content, duration) for each implementation <- graphics
		% Why these messages?
		% 3 emerged out of 1 and 2 (see results)
		% maybe a nice reference table for that too
		\lipsum[4]

	\section*{Case Study}
		% SETUP
		% * introduce prototypes
		%	* functionalities
		%
		% * introduce groups(+subgroups)
		%	* size
		%	* what do they do
		%	* naming conventions
		%
		% * introduction to participants
		%
		% * tasks (hints)
		%
		% * what do we measure and how? 
		%	* screen recordings
		%	* notepad
		%
		% * questionnaire
		%	* SUS
		%	* open question
		%
		%
		% RESULTS
		% * first proto toaster
		% * first proto microwave
		% * second proto toaster
		% * second proto microwave
		%
		% * did wrong once vs did wrong multiple times
		%
		% * time to fulfill task per level
		% * time to fulfill tasks per prototype order
		% * number of wrong usages per level
		% * percentage of which group asking for more feedback
		% 
		% DISCUSSION
		% * hypothesis "visible and better feedback msg help the user to fulfill tasks faster, easier"
		% * feedback message must be genera enough to be applicable to any possible situation
		% * feedback message must be specific enough to be more helpful in the given situation
		% * outlines: sus/free answers
		% * easy prototype microwave -> intuitively usable, no feedback needed
		%
		\lipsum[5]

	\section*{Outlook}
		\lipsum[6] \cite{Dey2016} \cite{Nilsson2007} \cite{Poupyrev2002} \ac{AR} \ac{AR}

	\pagebreak
	\printbibliography
\restoregeometry
\end{document}